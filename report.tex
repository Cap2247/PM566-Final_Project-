% Options for packages loaded elsewhere
\PassOptionsToPackage{unicode}{hyperref}
\PassOptionsToPackage{hyphens}{url}
%
\documentclass[
  12pt,
]{article}
\usepackage{amsmath,amssymb}
\usepackage{lmodern}
\usepackage{iftex}
\ifPDFTeX
  \usepackage[T1]{fontenc}
  \usepackage[utf8]{inputenc}
  \usepackage{textcomp} % provide euro and other symbols
\else % if luatex or xetex
  \usepackage{unicode-math}
  \defaultfontfeatures{Scale=MatchLowercase}
  \defaultfontfeatures[\rmfamily]{Ligatures=TeX,Scale=1}
\fi
% Use upquote if available, for straight quotes in verbatim environments
\IfFileExists{upquote.sty}{\usepackage{upquote}}{}
\IfFileExists{microtype.sty}{% use microtype if available
  \usepackage[]{microtype}
  \UseMicrotypeSet[protrusion]{basicmath} % disable protrusion for tt fonts
}{}
\usepackage{xcolor}
\usepackage[margin=1in]{geometry}
\usepackage{graphicx}
\makeatletter
\def\maxwidth{\ifdim\Gin@nat@width>\linewidth\linewidth\else\Gin@nat@width\fi}
\def\maxheight{\ifdim\Gin@nat@height>\textheight\textheight\else\Gin@nat@height\fi}
\makeatother
% Scale images if necessary, so that they will not overflow the page
% margins by default, and it is still possible to overwrite the defaults
% using explicit options in \includegraphics[width, height, ...]{}
\setkeys{Gin}{width=\maxwidth,height=\maxheight,keepaspectratio}
% Set default figure placement to htbp
\makeatletter
\def\fps@figure{htbp}
\makeatother
\setlength{\emergencystretch}{3em} % prevent overfull lines
\providecommand{\tightlist}{%
  \setlength{\itemsep}{0pt}\setlength{\parskip}{0pt}}
\setcounter{secnumdepth}{-\maxdimen} % remove section numbering
\usepackage{amsmath}
\usepackage{booktabs}
\usepackage{caption}
\usepackage{longtable}
\ifLuaTeX
  \usepackage{selnolig}  % disable illegal ligatures
\fi
\IfFileExists{bookmark.sty}{\usepackage{bookmark}}{\usepackage{hyperref}}
\IfFileExists{xurl.sty}{\usepackage{xurl}}{} % add URL line breaks if available
\urlstyle{same} % disable monospaced font for URLs
\hypersetup{
  pdftitle={Final Project},
  pdfauthor={CP},
  hidelinks,
  pdfcreator={LaTeX via pandoc}}

\title{Final Project}
\author{CP}
\date{2022-12-06}

\begin{document}
\maketitle

\hypertarget{introduction}{%
\section{\texorpdfstring{\textbf{Introduction}}{Introduction}}\label{introduction}}

\hypertarget{during-the-covid-19-pandemic-mental-health-services-were-disrupted-causing-profound-consequences-for-individuals-in-need-of-consistent-mental-health-care.-the-pandemic-exacerbated-the-need-for-mental-health-services-which-expanded-an-existing-unmet-demand.-during-the-pandemic-studies-have-found-the-need-for-mental-health-services-rose-from-one-in-ten-adults-requesting-services-for-depression-and-anxiety-to-one-in-four-adults-zhu-et-al.-2022.-this-supports-recent-findings-that-prolonged-periods-of-quarantine-social-isolation-and-work-and-school-disruption-are-associated-with-anxiety-and-psychological-distress-brooks-et-al.-2020.}{%
\paragraph{During the COVID-19 pandemic, mental health services were
disrupted, causing profound consequences for individuals in need of
consistent mental health care. The pandemic exacerbated the need for
mental health services which expanded an existing unmet demand. During
the pandemic, studies have found the need for mental health services
rose from one in ten adults requesting services for depression and
anxiety to one in four adults (Zhu et al., 2022). This supports recent
findings that prolonged periods of quarantine, social isolation, and
work and school disruption are associated with anxiety and psychological
distress (Brooks et al.,
2020).}\label{during-the-covid-19-pandemic-mental-health-services-were-disrupted-causing-profound-consequences-for-individuals-in-need-of-consistent-mental-health-care.-the-pandemic-exacerbated-the-need-for-mental-health-services-which-expanded-an-existing-unmet-demand.-during-the-pandemic-studies-have-found-the-need-for-mental-health-services-rose-from-one-in-ten-adults-requesting-services-for-depression-and-anxiety-to-one-in-four-adults-zhu-et-al.-2022.-this-supports-recent-findings-that-prolonged-periods-of-quarantine-social-isolation-and-work-and-school-disruption-are-associated-with-anxiety-and-psychological-distress-brooks-et-al.-2020.}}

\hypertarget{in-addition-to-a-shortage-of-mental-health-professionals-mental-health-providers-needed-to-adjust-service-delivery-to-comply-with-safety-protocols-which-further-inhibited-access-to-care-for-some-individuals.-to-meet-the-growing-demands-for-mental-health-care-the-us-government-waived-numerous-regulations-around-the-delivery-of-telemedicine-including-regulatory-barriers-that-would-allow-for-delivery-across-state-lines-bojdani-et-al.-2020.-however-after-surveying-32-californian-nonprofit-behavioral-health-agencies-bartels-et-al.-2020-found-that-87-lacked-the-equipment-required-to-facilitate-telehealth-care-and-of-the-health-agencies-that-could-continue-in-person-treatment-many-lacked-access-to-appropriate-ppe-for-staff.-additionally-states-had-to-reform-policies-around-payment-structures-to-allow-for-financial-support-of-remote-telemedicine-models.-despite-these-attempts-to-mitigate-the-unforeseen-challenges-to-mental-health-care-delivery-a-decrease-in-revenue-continued-to-occur-resulting-in-mental-health-professionals-being-furloughed-or-terminated-bojdani-et-al.-2020.}{%
\paragraph{In addition to a shortage of mental health professionals,
mental health providers needed to adjust service delivery to comply with
safety protocols, which further inhibited access to care for some
individuals. To meet the growing demands for mental health care, the US
government waived numerous regulations around the delivery of
telemedicine, including regulatory barriers that would allow for
delivery across state lines (Bojdani et al., 2020). However, after
surveying 32 Californian nonprofit behavioral health agencies, Bartels
et al.~(2020) found that 87\% lacked the equipment required to
facilitate telehealth care, and of the health agencies that could
continue in-person treatment, many lacked access to appropriate PPE for
staff. Additionally, states had to reform policies around payment
structures to allow for financial support of remote telemedicine models.
Despite these attempts to mitigate the unforeseen challenges to mental
health care delivery, a decrease in revenue continued to occur,
resulting in mental health professionals being furloughed or terminated
(Bojdani et al.,
2020).}\label{in-addition-to-a-shortage-of-mental-health-professionals-mental-health-providers-needed-to-adjust-service-delivery-to-comply-with-safety-protocols-which-further-inhibited-access-to-care-for-some-individuals.-to-meet-the-growing-demands-for-mental-health-care-the-us-government-waived-numerous-regulations-around-the-delivery-of-telemedicine-including-regulatory-barriers-that-would-allow-for-delivery-across-state-lines-bojdani-et-al.-2020.-however-after-surveying-32-californian-nonprofit-behavioral-health-agencies-bartels-et-al.-2020-found-that-87-lacked-the-equipment-required-to-facilitate-telehealth-care-and-of-the-health-agencies-that-could-continue-in-person-treatment-many-lacked-access-to-appropriate-ppe-for-staff.-additionally-states-had-to-reform-policies-around-payment-structures-to-allow-for-financial-support-of-remote-telemedicine-models.-despite-these-attempts-to-mitigate-the-unforeseen-challenges-to-mental-health-care-delivery-a-decrease-in-revenue-continued-to-occur-resulting-in-mental-health-professionals-being-furloughed-or-terminated-bojdani-et-al.-2020.}}

\hypertarget{at-the-onset-of-the-pandemic-several-state-and-local-governments-issued-stay-at-home-orders-to-reduce-viral-spread.-these-orders-resulted-in-increased-emotional-distress-e.g.-anxiety-depression-and-loss-of-employment.-fronstin-woodbury-2020-estimate-that-approximately-7.7-million-workers-and-6.9-million-dependants-lost-their-insurance-in-the-united-states-by-june-2020.-employers-are-the-leading-providers-of-health-insurance-in-the-united-states-suggesting-that-newly-unemployed-individuals-may-have-lost-access-to-secure-healthcare-coverage.-the-urban-institute-estimates-that-about-a-third-of-individuals-who-loose-employer-sponsored-insurance-will-become-uninsured-and-just-over-a-quarter-will-obtain-insurance-through-medicaid-or-the-childrens-health-insurance-program-chip.-considering-the-unprecedented-loss-of-employment-that-occurred-during-the-pandemic-the-shift-from-employer-sponsored-insurance-to-public-insurance-changed-how-individuals-were-accessing-necessary-mental-health-care.}{%
\paragraph{At the onset of the pandemic, several state and local
governments issued stay-at-home orders to reduce viral spread. These
orders resulted in increased emotional distress e.g., anxiety,
depression, and loss of employment. Fronstin \& Woodbury (2020) estimate
that approximately 7.7 million workers and 6.9 million dependants lost
their insurance in the United States by June 2020. Employers are the
leading providers of health insurance in the United States, suggesting
that newly unemployed individuals may have lost access to secure
healthcare coverage. The Urban Institute estimates that about a third of
individuals who loose employer-sponsored insurance will become
uninsured, and just over a quarter will obtain insurance through
Medicaid or the Children's Health Insurance Program (CHIP). Considering
the unprecedented loss of employment that occurred during the pandemic,
the shift from employer-sponsored insurance to public insurance changed
how individuals were accessing necessary mental health
care.}\label{at-the-onset-of-the-pandemic-several-state-and-local-governments-issued-stay-at-home-orders-to-reduce-viral-spread.-these-orders-resulted-in-increased-emotional-distress-e.g.-anxiety-depression-and-loss-of-employment.-fronstin-woodbury-2020-estimate-that-approximately-7.7-million-workers-and-6.9-million-dependants-lost-their-insurance-in-the-united-states-by-june-2020.-employers-are-the-leading-providers-of-health-insurance-in-the-united-states-suggesting-that-newly-unemployed-individuals-may-have-lost-access-to-secure-healthcare-coverage.-the-urban-institute-estimates-that-about-a-third-of-individuals-who-loose-employer-sponsored-insurance-will-become-uninsured-and-just-over-a-quarter-will-obtain-insurance-through-medicaid-or-the-childrens-health-insurance-program-chip.-considering-the-unprecedented-loss-of-employment-that-occurred-during-the-pandemic-the-shift-from-employer-sponsored-insurance-to-public-insurance-changed-how-individuals-were-accessing-necessary-mental-health-care.}}

\hypertarget{the-growing-demand-for-mental-health-services-possible-reduction-in-coverage-and-changing-delivery-systems-may-have-decreased-access-to-necessary-mental-health-services-during-the-covid-19-pandemic.-it-remains-unclear-the-extent-to-which-these-changes-in-the-mental-health-services-delivery-system-have-impacted-the-utilization-of-services.-moreover-it-is-unclear-how-insurance-influenced-access-of-mental-health-services.-therefore-the-primary-aim-of-this-study-is-to-establish-insurance-related-patterns-of-mental-healthcare-utilization-from-october-28-2020-november-9-2020.-specifically-did-the-type-of-health-insurance-affect-access-to-mental-health-services-during-the-covid-19-pandemic}{%
\paragraph{The growing demand for mental health services, possible
reduction in coverage, and changing delivery systems, may have decreased
access to necessary mental health services during the COVID-19 pandemic.
It remains unclear the extent to which these changes in the mental
health services delivery system have impacted the utilization of
services. Moreover, it is unclear how insurance influenced access of
mental health services. Therefore, the primary aim of this study is to
establish insurance-related patterns of mental healthcare utilization
from October 28, 2020 -- November 9, 2020. Specifically, did the type of
health insurance affect access to mental health services during the
COVID-19
pandemic?}\label{the-growing-demand-for-mental-health-services-possible-reduction-in-coverage-and-changing-delivery-systems-may-have-decreased-access-to-necessary-mental-health-services-during-the-covid-19-pandemic.-it-remains-unclear-the-extent-to-which-these-changes-in-the-mental-health-services-delivery-system-have-impacted-the-utilization-of-services.-moreover-it-is-unclear-how-insurance-influenced-access-of-mental-health-services.-therefore-the-primary-aim-of-this-study-is-to-establish-insurance-related-patterns-of-mental-healthcare-utilization-from-october-28-2020-november-9-2020.-specifically-did-the-type-of-health-insurance-affect-access-to-mental-health-services-during-the-covid-19-pandemic}}

\hypertarget{methods}{%
\section{\texorpdfstring{\textbf{Methods}}{Methods}}\label{methods}}

\hypertarget{data}{%
\subsection{Data}\label{data}}

\hypertarget{we-conducted-a-retrospective-cross-sectional-analysis-of-the-u.s.-census-bureaus-household-pulse-survey-hps-covid-19-data.-we-choose-to-evaluate-data-from-the-phase-3-data-release-week-18-october-28-2020--november-9-2020.-the-hps-is-a-20-min-online-survey-designed-to-quickly-and-efficiently-measure-the-social-and-economic-impacts-of-covid-19-on-american-households.-the-census-bureau-releases-the-hps-data-every-4-weeks.}{%
\paragraph{We conducted a retrospective cross-sectional analysis of the
U.S. Census Bureau's Household Pulse Survey (HPS) COVID-19 data. We
choose to evaluate data from the phase 3 data release, week 18 October
28, 2020- November 9, 2020. The HPS is a 20 min online survey designed
to quickly and efficiently measure the social and economic impacts of
COVID-19 on American households. The Census Bureau releases the HPS data
every 4
weeks.}\label{we-conducted-a-retrospective-cross-sectional-analysis-of-the-u.s.-census-bureaus-household-pulse-survey-hps-covid-19-data.-we-choose-to-evaluate-data-from-the-phase-3-data-release-week-18-october-28-2020--november-9-2020.-the-hps-is-a-20-min-online-survey-designed-to-quickly-and-efficiently-measure-the-social-and-economic-impacts-of-covid-19-on-american-households.-the-census-bureau-releases-the-hps-data-every-4-weeks.}}

\hypertarget{variable-identification-and-analysis}{%
\subsection{Variable identification and
analysis}\label{variable-identification-and-analysis}}

\hypertarget{our-sampling-frame-was-all-individuals-who-requested-mental-health-services-and-either-received-services-or-did-not-between-october-28-2021-and-november-9-2021-n-58729.-we-evaluated-utilization-of-services-as-mental-health-services-counseling-or-therapy-were-requested-but-not-received-from-a-mental-health-professional-at-any-time-in-the-past-four-weeks.-variables-of-interest-for-this-study-included-public-insurance-all-individuals-with-medicare-medicaid-and-veterans-affairs-va-health-insurance-private-insurance-employer-sponsored-health-insurance-insurance-purchased-from-a-health-insurance-agency-tricare-or-other-military-health-insurance-and-income-total-household-income-before-taxes-in-2019.}{%
\paragraph{Our sampling frame was all individuals who requested mental
health services and either received services or did not between October
28, 2021, and November 9, 2021 (N= 58,729). We evaluated utilization of
services as ``mental health services (counseling or therapy) were
requested but not received from a mental health professional at any time
in the past four weeks''. Variables of interest for this study included
public insurance (all individuals with Medicare, Medicaid, and Veterans
Affairs (VA) health insurance), private insurance (employer-sponsored
health insurance, insurance purchased from a health insurance agency,
TRICARE or other military health insurance) and income (total household
income before taxes in
2019).}\label{our-sampling-frame-was-all-individuals-who-requested-mental-health-services-and-either-received-services-or-did-not-between-october-28-2021-and-november-9-2021-n-58729.-we-evaluated-utilization-of-services-as-mental-health-services-counseling-or-therapy-were-requested-but-not-received-from-a-mental-health-professional-at-any-time-in-the-past-four-weeks.-variables-of-interest-for-this-study-included-public-insurance-all-individuals-with-medicare-medicaid-and-veterans-affairs-va-health-insurance-private-insurance-employer-sponsored-health-insurance-insurance-purchased-from-a-health-insurance-agency-tricare-or-other-military-health-insurance-and-income-total-household-income-before-taxes-in-2019.}}

\hypertarget{we-conducted-descriptive-analyses-comparing-the-utilization-count-of-mental-health-services-by-insurance-type.-we-compared-the-rates-of-receiving-mental-health-services-between-public-and-private-insurance-by-state-and-income.-there-was-a-total-of-189510-missing-variables-that-were-excluded-from-this-analysis.}{%
\paragraph{We conducted descriptive analyses comparing the utilization
count of mental health services by insurance type. We compared the rates
of receiving mental health services between public and private insurance
by state and income. There was a total of 189,510 missing variables that
were excluded from this
analysis.}\label{we-conducted-descriptive-analyses-comparing-the-utilization-count-of-mental-health-services-by-insurance-type.-we-compared-the-rates-of-receiving-mental-health-services-between-public-and-private-insurance-by-state-and-income.-there-was-a-total-of-189510-missing-variables-that-were-excluded-from-this-analysis.}}

\hypertarget{results}{%
\section{\texorpdfstring{\textbf{Results}}{Results}}\label{results}}

\hypertarget{please-see-study-webpage-for-interactive-figures}{%
\subparagraph{Please see study webpage for interactive
figures}\label{please-see-study-webpage-for-interactive-figures}}

\hypertarget{demographic-characteristics-for-our-study-sample-by-region-are-presented-in-table-1.-our-final-analysis-consisted-of-39778-individuals-from-four-regions-within-the-united-states.-individuals-receiving-between-100-150k-a-year-represent-19-of-our-sample.-individuals-with-private-insurance-make-up-over-50-of-our-sample.}{%
\paragraph{Demographic characteristics for our study sample by region
are presented in Table 1. Our final analysis consisted of 39,778
individuals from four regions within the United States. Individuals
receiving between 100-150k a year represent 19\% of our sample.
Individuals with private insurance make up over 50\% of our
sample.}\label{demographic-characteristics-for-our-study-sample-by-region-are-presented-in-table-1.-our-final-analysis-consisted-of-39778-individuals-from-four-regions-within-the-united-states.-individuals-receiving-between-100-150k-a-year-represent-19-of-our-sample.-individuals-with-private-insurance-make-up-over-50-of-our-sample.}}

\setlength{\LTpost}{0mm}
\begin{longtable}{lc}
\caption*{
{\large Table 1. Demographic Characterisitics}
} \\ 
\toprule
\textbf{Characteristic} & \textbf{N = 39,778}\textsuperscript{1} \\ 
\midrule
Region &  \\ 
    Midwest & 8,272 (21\%) \\ 
    Northeast & 5,824 (15\%) \\ 
    South & 11,918 (30\%) \\ 
    West & 13,764 (35\%) \\ 
Income &  \\ 
     25K and below & 3,947 (9.9\%) \\ 
    100K - 149.9K & 7,456 (19\%) \\ 
    150K - 199.9K & 3,603 (9.1\%) \\ 
    199.9K and above & 4,187 (11\%) \\ 
    25K - 34.9K & 3,337 (8.4\%) \\ 
    35K - 49.9K & 4,339 (11\%) \\ 
    50K - 74.9K & 6,979 (18\%) \\ 
    75K - 99.9K & 5,930 (15\%) \\ 
Insurance\_type &  \\ 
    Both & 9,613 (24\%) \\ 
    No Insurance & 2,641 (6.6\%) \\ 
    Private insurance only & 21,890 (55\%) \\ 
    Public insurance only & 5,634 (14\%) \\ 
\bottomrule
\end{longtable}
\begin{minipage}{\linewidth}
\textsuperscript{1}n (\%)\\
\end{minipage}

\hypertarget{figure-1.-the-bar-graph-illustrates-the-count-and-rate-of-mental-health-services-requested-and-not-received-in-eight-income-categories.-among-those-with-public-insurance-individuals-making-less-than-25k-a-year-have-the-highest-rate-of-not-receiving-services-19.7-compared-to-other-income-levels.-although-individuals-with-private-insurance-who-make-between-100k-and-150k-a-year-have-the-highest-count-of-services-requested-and-not-received-individual-making-less-than-25k-and-up-to-34.9k-have-a-higher-rate-of-not-receiving-services-18.92-18.91-respectively.}{%
\paragraph{Figure 1. The bar graph illustrates the count and rate of
mental health services requested and not received in eight income
categories. Among those with public insurance, individuals making less
than 25k a year have the highest rate of not receiving services (19.7
\%) compared to other income levels. Although individuals with private
insurance who make between 100k and 150k a year have the highest count
of services requested and not received, individual making less than 25k
and up to 34.9k have a higher rate of not receiving services (18.92\%,
18.91\%
respectively).}\label{figure-1.-the-bar-graph-illustrates-the-count-and-rate-of-mental-health-services-requested-and-not-received-in-eight-income-categories.-among-those-with-public-insurance-individuals-making-less-than-25k-a-year-have-the-highest-rate-of-not-receiving-services-19.7-compared-to-other-income-levels.-although-individuals-with-private-insurance-who-make-between-100k-and-150k-a-year-have-the-highest-count-of-services-requested-and-not-received-individual-making-less-than-25k-and-up-to-34.9k-have-a-higher-rate-of-not-receiving-services-18.92-18.91-respectively.}}

\hypertarget{figure-2.-this-scatter-plot-illustrates-the-number-of-services-that-were-requested-and-not-received-during-october-28-2021-november-9-2021.-it-can-be-seen-that-california-has-the-highest-number-of-mental-health-services-that-were-requested-and-not-received-during-the-pandemic-among-those-with-private-insurance-1678-requests-227-not-received.}{%
\paragraph{Figure 2. This scatter plot illustrates the number of
services that were requested and not received during October 28, 2021 --
November 9, 2021. It can be seen that California has the highest number
of mental health services that were requested and not received during
the pandemic among those with private insurance (1,678 requests, 227 not
received).}\label{figure-2.-this-scatter-plot-illustrates-the-number-of-services-that-were-requested-and-not-received-during-october-28-2021-november-9-2021.-it-can-be-seen-that-california-has-the-highest-number-of-mental-health-services-that-were-requested-and-not-received-during-the-pandemic-among-those-with-private-insurance-1678-requests-227-not-received.}}

\hypertarget{figure-3.-this-scatter-plot-illustrates-the-number-of-services-that-were-requested-and-not-received-during-october-28-2021-november-9-2021.-it-can-be-seen-that-california-has-the-highest-number-of-mental-health-services-that-were-requested-and-not-received-during-the-pandemic-among-those-with-public-insurance-440-requested-64-not-received-.}{%
\paragraph{Figure 3. This scatter plot illustrates the number of
services that were requested and not received during October 28, 2021 --
November 9, 2021. It can be seen that California has the highest number
of mental health services that were requested and not received during
the pandemic among those with public insurance (440 requested, 64 not
received
).}\label{figure-3.-this-scatter-plot-illustrates-the-number-of-services-that-were-requested-and-not-received-during-october-28-2021-november-9-2021.-it-can-be-seen-that-california-has-the-highest-number-of-mental-health-services-that-were-requested-and-not-received-during-the-pandemic-among-those-with-public-insurance-440-requested-64-not-received-.}}

\hypertarget{figure-4.-this-map-depicts-the-total-rate-of-mental-health-services-requested-and-not-received-for-individuals-with-both-public-and-private-insurance-by-state.-although-california-had-the-highest-count-of-services-not-received-among-private-and-public-insurance-respectively-california-has-a-lower-rate-of-services-not-received-13.7-than-utah-14.5-oregon-15.8-oklahoma-16.2-and-vermont-14.6.}{%
\paragraph{Figure 4. This map depicts the total rate of mental health
services requested and not received for individuals with both public and
private insurance by state. Although California had the highest count of
services not received among private and public insurance, respectively,
California has a lower rate of services not received (13.7\%) than Utah
(14.5\%), Oregon (15.8\%), Oklahoma (16.2\%), and Vermont
(14.6\%).}\label{figure-4.-this-map-depicts-the-total-rate-of-mental-health-services-requested-and-not-received-for-individuals-with-both-public-and-private-insurance-by-state.-although-california-had-the-highest-count-of-services-not-received-among-private-and-public-insurance-respectively-california-has-a-lower-rate-of-services-not-received-13.7-than-utah-14.5-oregon-15.8-oklahoma-16.2-and-vermont-14.6.}}

\hypertarget{conclusion-and-summary}{%
\section{\texorpdfstring{\textbf{Conclusion and
Summary}}{Conclusion and Summary}}\label{conclusion-and-summary}}

\hypertarget{this-study-specifically-aimed-to-evaluate-differences-in-mental-health-services-utilization-among-individuals-with-private-or-public-insurance-during-the-early-months-of-the-coviid-19-pandemic.-we-have-found-that-overall-individuals-with-private-insurance-have-higher-rates-of-not-receiving-services-than-those-who-have-public-insurance.-this-is-a-surprising-finding-considering-mechanisms-within-private-insurance-have-been-thought-to-make-healthcare-more-accessible-than-the-systemic-bureaucracy-of-medicare-and-medicaid.-these-results-indicate-the-benefits-that-recent-changes-to-medicaid-may-have-made-to-increase-access-to-services.}{%
\paragraph{This study specifically aimed to evaluate differences in
mental health services utilization among individuals with private or
public insurance during the early months of the COVIID-19 pandemic. We
have found that overall, individuals with private insurance have higher
rates of not receiving services than those who have public insurance.
This is a surprising finding considering mechanisms within private
insurance have been thought to make healthcare more accessible than the
systemic bureaucracy of Medicare and Medicaid. These results indicate
the benefits that recent changes to Medicaid may have made to increase
access to
services.}\label{this-study-specifically-aimed-to-evaluate-differences-in-mental-health-services-utilization-among-individuals-with-private-or-public-insurance-during-the-early-months-of-the-coviid-19-pandemic.-we-have-found-that-overall-individuals-with-private-insurance-have-higher-rates-of-not-receiving-services-than-those-who-have-public-insurance.-this-is-a-surprising-finding-considering-mechanisms-within-private-insurance-have-been-thought-to-make-healthcare-more-accessible-than-the-systemic-bureaucracy-of-medicare-and-medicaid.-these-results-indicate-the-benefits-that-recent-changes-to-medicaid-may-have-made-to-increase-access-to-services.}}

\hypertarget{additionally-we-found-that-individuals-who-make-less-than-25000-a-year-have-similarly-high-rates-of-not-receiving-services-for-both-private-and-public-insurance.-this-finding-illustrates-the-disparities-that-continue-to-exist-for-individuals-whose-income-is-below-the-poverty-line.-within-this-income-bracket-private-insurance-a-proxy-for-increased-socioeconomic-status-is-not-as-protective-against-the-social-barriers-to-healthcare-as-it-is-for-individuals-in-increased-income-ranges.-further-research-is-needed-to-identify-service-and-diagnosis-specific-patterns-in-mental-health-service-utilization.}{%
\paragraph{Additionally, we found that individuals who make less than
\$25,000 a year have similarly high rates of not receiving services for
both private and public insurance. This finding illustrates the
disparities that continue to exist for individuals whose income is below
the poverty line. Within this income bracket, private insurance, a proxy
for increased socioeconomic status, is not as protective against the
social barriers to healthcare as it is for individuals in increased
income ranges. Further research is needed to identify service and
diagnosis-specific patterns in mental health service
utilization.}\label{additionally-we-found-that-individuals-who-make-less-than-25000-a-year-have-similarly-high-rates-of-not-receiving-services-for-both-private-and-public-insurance.-this-finding-illustrates-the-disparities-that-continue-to-exist-for-individuals-whose-income-is-below-the-poverty-line.-within-this-income-bracket-private-insurance-a-proxy-for-increased-socioeconomic-status-is-not-as-protective-against-the-social-barriers-to-healthcare-as-it-is-for-individuals-in-increased-income-ranges.-further-research-is-needed-to-identify-service-and-diagnosis-specific-patterns-in-mental-health-service-utilization.}}

\newpage

\hypertarget{references}{%
\section{\texorpdfstring{\textbf{References}}{References}}\label{references}}

\hypertarget{bartels-s.-j.-baggett-t.-p.-freudenreich-o.-bird-b.-l.-2020.-covid-19-emergency-reforms-in-massachusetts-to-support-behavioral-health-care-and-reduce-mortality-of-people-with-serious-mental-illness.-psychiatric-services-7110-1078-1081}{%
\paragraph{Bartels, S. J., Baggett, T. P., Freudenreich, O., \& Bird, B.
L. (2020). COVID-19 emergency reforms in Massachusetts to support
behavioral health care and reduce mortality of people with serious
mental illness. Psychiatric Services, 71(10),
1078-1081}\label{bartels-s.-j.-baggett-t.-p.-freudenreich-o.-bird-b.-l.-2020.-covid-19-emergency-reforms-in-massachusetts-to-support-behavioral-health-care-and-reduce-mortality-of-people-with-serious-mental-illness.-psychiatric-services-7110-1078-1081}}

\hypertarget{bojdani-e.-rajagopalan-a.-chen-a.-gearin-p.-olcott-w.-shankar-v.-delisi-l.-e.-2020.-covid-19-pandemic-impact-on-psychiatric-care-in-the-united-states.-psychiatry-research-289-113069.}{%
\paragraph{Bojdani, E., Rajagopalan, A., Chen, A., Gearin, P., Olcott,
W., Shankar, V., \ldots{} \& DeLisi, L. E. (2020). COVID-19 pandemic:
impact on psychiatric care in the United States. Psychiatry research,
289,
113069.}\label{bojdani-e.-rajagopalan-a.-chen-a.-gearin-p.-olcott-w.-shankar-v.-delisi-l.-e.-2020.-covid-19-pandemic-impact-on-psychiatric-care-in-the-united-states.-psychiatry-research-289-113069.}}

\hypertarget{brooks-sk-webster-rk-smith-le-woodland-l-wessely-s-greenberg-n-et-al.-the-psychological-impact-of-quarantine-and-how-to-reduce-it-rapid-review-of-the-evidence.-lancet.-20203951022791220.-pubmed-32112714}{%
\paragraph{Brooks SK, Webster RK, Smith LE, Woodland L, Wessely S,
Greenberg N, et al.~The psychological impact of quarantine and how to
reduce it: rapid review of the evidence. Lancet.
2020;395(10227):912--20. {[}PubMed:
32112714{]}}\label{brooks-sk-webster-rk-smith-le-woodland-l-wessely-s-greenberg-n-et-al.-the-psychological-impact-of-quarantine-and-how-to-reduce-it-rapid-review-of-the-evidence.-lancet.-20203951022791220.-pubmed-32112714}}

\hypertarget{fronstin-p.-woodbury-s.-a.-2020.-how-many-americans-have-lost-jobs-with-employer-health-coverage-during-the-pandemic.}{%
\paragraph{Fronstin, P., \& Woodbury, S. A. (2020). How many Americans
have lost jobs with employer health coverage during the
pandemic?.}\label{fronstin-p.-woodbury-s.-a.-2020.-how-many-americans-have-lost-jobs-with-employer-health-coverage-during-the-pandemic.}}

\hypertarget{zhu-j.-m.-myers-r.-mcconnell-k.-j.-levander-x.-lin-s.-c.-2022.-trends-in-outpatient-mental-health-services-use-before-and-during-the-covid-19-pandemic-study-examines-trends-in-outpatient-mental-health-service-using-before-and-during-the-covid-19-pandemic.-health-affairs-414-573-580.}{%
\paragraph{Zhu, J. M., Myers, R., McConnell, K. J., Levander, X., \&
Lin, S. C. (2022). Trends In Outpatient Mental Health Services Use
Before And During The COVID-19 Pandemic: Study examines trends in
outpatient mental health service using before and during the COVID-19
pandemic. Health Affairs, 41(4),
573-580.}\label{zhu-j.-m.-myers-r.-mcconnell-k.-j.-levander-x.-lin-s.-c.-2022.-trends-in-outpatient-mental-health-services-use-before-and-during-the-covid-19-pandemic-study-examines-trends-in-outpatient-mental-health-service-using-before-and-during-the-covid-19-pandemic.-health-affairs-414-573-580.}}

\textbackslash end\{document\}

\end{document}
